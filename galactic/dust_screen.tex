\documentclass{standalone}
%\pagenumbering{gobble}


%%%%%%%%%%%%%%%%%%%%%%%%%%%%%%%%%%%%%%%%%%%%
% Dust screen method for determining
% distances to clouds
% (based on Figure 2 of Zucker+ 2019)
%
% random points code snippet from
% https://tex.stackexchange.com/questions/145969/filling-specified-area-by-random-dots-in-tikz
%
%%%%%%%%%%%%%%%%%%%%%%%%%%%%%%%%%%%%%%%%%%%%


\usepackage{tikz,xcolor}
\usepackage{pgfplots}
\usetikzlibrary{shadings, patterns}

\begin{document}

%\tikzstyle{background grid}=[draw, black!30,step=.5cm]
%\begin{tikzpicture}[scale=1.0, font=\sffamily, show background grid]
\begin{tikzpicture}
[
	scale=1.0, font=\sffamily,
    declare function={a(\x)=0.05*\x;},
]

	% shaded regions
	\begin{scope}
		\shade[shading=axis,bottom color=white,top color=white,middle color = gray!30, shading angle=10] (0,0) rectangle (3.7,1.5);
		\shade[shading=axis,bottom color=white,top color=white,middle color = gray!50, shading angle=0] (4,2.5) rectangle (8.5,5);
	\end{scope}
	\draw[thick, <->] (4.1,0.85) -- (4.1,3.75);
	\node at (4.6, 2.25) {$\Delta A_{\rm V}$};

	% draw random points
	\draw plot [only marks, mark=*, mark size=0.5, domain=0.3:3.5, samples=75] (\x,{rnd*0.8+0.3+0.05*\x});
	\draw plot [only marks, mark=*, mark size=0.5, domain=4.2:8.3, samples=200] (\x,{rnd*1.6+2.9});


	% cloud
	\draw[pattern=north west lines, draw=none] (3.71,0) rectangle (3.99,5);

	% axes
	\draw[thick, ->] (0,0) -- (8.5,0);
	\draw[thick, ->] (0,0) -- (0,5);
	\node at (3.85, -0.25) {$d_{\rm cloud}$};
	\node at (7.7, -0.25) {Distance};
	\node[rotate=90] at (-0.3, 4.2) {Extinction};


\end{tikzpicture}

\end{document} 
